%        File: bsr_wise.tex
%     Created: Sun Apr 20 11:00 AM 2014 P
% Last Change: Sun Apr 20 11:00 AM 2014 P
%
\documentclass[letterpaper]{article}
\usepackage[top=1.0in,bottom=1.0in,left=1.0in,right=1.0in]{geometry}
\usepackage{verbatim}
\usepackage{amssymb}
\usepackage{graphicx}
\usepackage{longtable}
\usepackage{amsfonts}
\usepackage{amsmath}
\usepackage[usenames]{color}
\usepackage[
  naturalnames = true, 
  colorlinks = true, 
  linkcolor = Black,
  anchorcolor = Black,
  citecolor = Black,
  menucolor = Black,
  urlcolor = Blue
]{hyperref}
\def\thesection       {\arabic{section}}
\def\thesubsection     {\thesection.\alph{subsection}}

\author{Dr. Kathryn D. Huff
  \\ \href{mailto:huff@berkeley.edu}{\texttt{huff@berkeley.edu}}
}

\date{}
\title{Wise as Athena, Swifter than Hermes:\\
Wonderwomen Gather to Improve Scientific Computing}
\begin{document}
\maketitle

\section*{The Future of Science}

%Computational skills are essential to nearly every profession today. Just as 
%baristas and delivery drivers rely on sophisticated handheld 
%tablets, 
Scientific discovery is increasingly reliant on 
software and computation. Database interfaces and content 
management systems have replaced the lab notebook, simulations are replacing 
pencil-and-paper analytics, and nights once spent leaning over an experimental 
benchtop are now spent complemented by days spent glued to a computer screen. 
But, we're doing it wrong. 

The dream, of course, is that computers should empower scientists. Computers 
should help us produce more robust analyses, arrive at more accurate results, 
and make more discoveries. 
The reality, however, is that research science is far from that dream 
\cite{hatton_t_1997,merali_computational_2010,joppa_troubling_2013} .  
Scientists are often frustrated by data management, are unaware of code testing 
best practices, and struggle to collaborate in the absence of modern software 
development workflows 
\cite{ackroyd_scientific_2008,segal_when_2005,hannay_how_2009,dubois_why_2003}.  
Accordingly, journal articles are too often retracted or due to software bugs that 
went undetected in the review process \cite{hypertension2012,chang2007,jaccretract2013,miller2006}.


What with its `big data,' intricate numerics, and 
fundamental philosophies of accuracy and reproducibilty,  scientific research 
should be replete with modern software development techniques and best 
practices.  Computation is clearly so integral to the future of science that 
scientific software, combined with version control, robust testing, automated 
documentation, and scripted analysis, might someday be a research object 
equally, or, if we can be sufficiently bold, vastly more important than the 
almighty journal article.

Reproducibility of scientific software suffers in this way for many reasons. For 
example, there is no room in university curriculum to adequately train 
scientists in computational and numerical skills that their research will rely 
on. Furthermore, academic science is built on an incentive structure that fails 
to reward effort spent on software robustness. Appropriately, scientists who do 
somehow acquire computational skill are drawn away from academic research by an 
industry to reward them for those skills \cite{vanderplas_big_2013}.  

Working to bring reality closer to the dream, a non-profit organization called 
Software Carpentry \cite{wilson_software_2014} seeks to spread awareness and 
essential skills for scientific computation. At Lawrence Berkeley National 
Laboratory last week, the future of science began to look a little brighter, as 
approximately 140 scientists and volunteer experts gathered for a Software 
Carpentry bootcamp to learn and teach the skills necessary to responsibly 
develop scientific software \cite{huff_software_2014}.

\section*{Wonderwomen, Changing the Ratio}
The first person to write an algorithm \cite{toole_ada_1992}, the first people to hold 
the job title ``computer,'' the first people to program the 
ENIAC \cite{grier_when_2013}, and all of the attendees of last week's bootcamp have 
something in common. They were all women.  

Despite a decades-long, hopeful trend toward gender equality among STEM disciplines, 
(a modest increase in the number of women in most STEM fields), 
computer science has seen a steady decline in female participation 
\cite{nsf_women_2013}.
Accordingly, professions in software development and other forms of computing 
are overwhelmingly male today \cite{margolis_unlocking_2003}.

The result is that science is increasingly reliant on a discipline dominated by 
men: computation.  Almost every domain now boasts a speciality involving 
computing (Computational Physics, Computational Neuroscience, Bioinformatics, 
Quantitative Ecology - the list goes on). Unfortunately, in the same way that 
the sciences and computing have struggled to attract, include, and retain women, 
these specialized, computational subfields of the sciences sometimes struggle 
twofold.  

Last weeks' bootcamp at LBNL was taught and attended entirely by females in the 
sciences in an effort to help counteract that struggle. These women, including 
Bay Area students, faculty, laboratory staff, and industry scientists, gathered 
for a two-day course on scientific computing best practices.  An all-female cast 
of seven volunteer instructors was on a mission to help scientists and engineers 
become more productive by teaching them basic lab skills for computing like 
program design, version control, data management, and task automation.  These 
instructors taught topics such as the bash shell, python, and git 
Over a dozen volunteer helpers of both genders also came to support the learning 
process.

This course was one of the largest bootcamps ever conducted by Software 
Carpentry. Most Software Carpentry bootcamps are co-ed, and many require an 
admission fee for attendees, to cover costs.  Howevever, this bootcamp is the 
second female-only bootcamp (the first was in Boston last year) 
and Software Carpentry was able to offer it for free thanks to a number of 
industry sponsors.  The women who registered came from a mix of scientific 
and engineering discplines and a range of skill levels. All of the attendees 
were familiar with basic programming concepts (like loops, conditionals, arrays, 
and functions) but sought to translate this knowledge into 
practical tools for working more productively.

Though a large number of these scientists are within a small female minority in 
their disciplines and correspondingly often work in majority-male enviornments, 
the feedback from the bootcamp indicated that the female-only learning 
environment was an especially enjoyable learning experinece for many attendees.  
Many attendees felt empowered and remarked on the collaborative nature of the 
classroom dynamic. 

Science cannot be called science if it is not robustly tested, reproducible, and 
transparent. But, in scientific research, computational analysis is too often 
treated as a black box. 



\bibliographystyle{ieeetr}
\bibliography{bsr_wise}

\end{document}


