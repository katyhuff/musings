%        File: bsr_wise.tex
%     Created: Sun Apr 20 11:00 AM 2014 P
% Last Change: Sun Apr 20 11:00 AM 2014 P
%
\documentclass[letterpaper]{article}
\usepackage[top=1.0in,bottom=1.0in,left=1.0in,right=1.0in]{geometry}
\usepackage{verbatim}
\usepackage{amssymb}
\usepackage{graphicx}
\usepackage{longtable}
\usepackage{amsfonts}
\usepackage{amsmath}
\usepackage[usenames]{color}
\usepackage[
  naturalnames = true, 
  colorlinks = true, 
  linkcolor = Black,
  anchorcolor = Black,
  citecolor = Black,
  menucolor = Black,
  urlcolor = Blue
]{hyperref}
\def\thesection       {\arabic{section}}
\def\thesubsection     {\thesection.\alph{subsection}}

\author{Dr. Kathryn D. Huff
  \\ \href{mailto:huff@berkeley.edu}{\texttt{huff@berkeley.edu}}
}

\date{}
\title{Wise as Athena, Swifter than Hermes:\\
Wonderwomen Gather to Improve Scientific Computing}
\begin{document}
\maketitle

\section*{The Computational Problem in Science}

%Computational skills are essential to nearly every profession today. Just as 
%baristas and delivery drivers rely on sophisticated handheld 
%tablets, 
Scientific discovery increasingly relies on 
software and computation. Database interfaces and content 
management systems have replaced the lab notebook, simulations are replacing 
pencil-and-paper analytics, and nights once spent leaning over an experimental 
benchtop are now spent complemented by days spent glued to a computer screen. 
But, we're doing it wrong. 

The dream, of course, is that computers should empower scientists. Computers 
should help us produce more robust analyses, arrive at more accurate results, 
and make more discoveries. 
The reality, however, is that research science is far from that dream 
\cite{hatton_t_1997,merali_computational_2010,joppa_troubling_2013} .  
Scientists are often frustrated by data management, are unaware of code testing 
best practices, and struggle to collaborate in the absence of modern software 
development workflows 
\cite{ackroyd_scientific_2008,segal_when_2005,hannay_how_2009,dubois_why_2003}.  
Accordingly, software bugs, undetected in the review process, have recently 
caused many high-profile journal article retractions 
\cite{hypertension2012,chang2007,jaccretract2013,miller2006}.

What with its `big data,' intricate numerics, and fundamental philosophies of 
accuracy and reproducibilty,  scientific research should be replete with modern 
software development techniques and best practices. However, reproducibility of 
scientific software suffers for many reasons. For example, there is no room in 
university curriculum to adequately train scientists in computational and 
numerical skills that their research will rely on. Furthermore, the incentive 
structure behind academic science fails to reward effort spent on software 
robustness, testing, and provenance-driven workflows. Appropriately, scientists 
who do somehow acquire computational skill often leave academic research 
to industry, where rewards for those skills are handsome \cite{vanderplas_big_2013}.  


%Computation is clearly so integral to the future of science that 
%scientific software, combined with version control, robust testing, automated 
%documentation, and scripted analysis, might someday be a research object 
%equally, or, if we can be sufficiently bold, vastly more important than the 
%almighty journal article.

\section*{Wonderwomen, To the Rescue}

Working to bring reality closer to the dream, a non-profit organization, 
\href{http://software-carpentry.org}{Software Carpentry} 
\cite{wilson_software_2014}, hosts bootcamps that spread awareness of and teach 
skills for scientific computation. Two weeks ago, at Lawrence Berkeley National 
Laboratory, the future of science began to look a little brighter as 
approximately 140 scientists and volunteer experts gathered for a two-day 
Software Carpentry bootcamp that was 
\href{http://swcarpentry.github.io/2014-04-14-wise}{particularly special} 
\cite{huff_software_2014}.  

The first person to write an algorithm \cite{toole_ada_1992}, the first people 
to hold the job title ``computer,'' the first people to program the ENIAC 
\cite{grier_when_2013}, and all of the attendees of last week's bootcamp have 
something in common. They were all women.  

Despite a decades-long, hopeful trend toward gender equality among STEM disciplines, 
(a modest increase in the number of women in most STEM fields), 
computer science has seen a steady decline in female participation 
\cite{nsf_women_2013}.
%Accordingly, professions in software development and other forms of computing 
%are overwhelmingly male today \cite{margolis_unlocking_2003}.

The result is that science is increasingly reliant on a discipline dominated by 
men: computation.  Almost every domain now boasts a speciality involving 
computing (Computational Physics, Computational Neuroscience, Bioinformatics, 
Quantitative Ecology - the list goes on). Unfortunately, in the same way that 
the sciences and computing have struggled to attract, include, and retain women, 
these specialized, computational subfields of the sciences sometimes struggle 
twofold.  

The Women in Science and Engineering (WiSE) Software Carpentry bootcamp at LBNL 
was taught and attended entirely by females in the sciences in an effort to help 
counteract that struggle.  Most Software Carpentry bootcamps are co-ed, but this 
bootcamp was the second in a series of female-focused bootcamps. The women who 
registered came from a mix of scientific and engineering discplines and a range 
of skill levels.  They included Bay Area students, faculty, laboratory staff, 
and industry scientists.

The all-female cast of seven volunteer instructors taught basic lab skills for 
computing like program design, version control, data management, and task 
automation.  As the lead instructor, it was my great pleasure to work with local 
expert instructors Cindee Madison, Suzanne Kiihne, and Professor Rachel 
Slaybaugh as well as experts from accross the US, Azalee Bostroem, Jessica Kerr, 
and Molly Gibson.  Over a dozen volunteer helpers of both genders also came to 
support the learning process as students were immersed in topics such as the 
bash shell, python, and git.  

The bootcamp seems to have been a resounding success. On the scientific quality front, 
attendees reported that they felt empowered to conduct their science more 
reproducibly after the bootcamp. And, on the gender equality front, many 
commented that the female-only learning environment contributed powerfully to an 
enjoyable learning experience. Wonderwomen - to the rescue!

\bibliographystyle{ieeetr}
\bibliography{bsr_wise}

\end{document}


